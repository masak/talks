\documentclass{beamer}

\usepackage[utf8]{inputenc}

\begin{document}
\title{A Perl 6 trek for Perl 5 pilgrims}
\author{Carl Mäsak}
\date{\today}

\frame{\titlepage}

\frame{\frametitle{Table of contents}\tableofcontents}

\section{A crash course in Perl 6}

\begin{frame}[fragile]
\frametitle{Perl 6 is still Perl, but different}
\begin{verbatim}
say 'Hello, Perl 5 people!';
\end{verbatim}
\end{frame}

\begin{frame}[fragile]
\frametitle{New operators}
\begin{verbatim}
# bitwise operators
5   +| 3;       # 7        # ternary op
6   +^ 3        # 6        $a == $b ?? 2 * $a !! $b - $a
5   +& 3;       # 1
"b" ~| "d"      # 'f'

# string concatenation
'a' ~ 'b'       # 'ab'

# file tests
if '/etc/passwd' ~~ :e { say "exists" }

# repetition
'a' x 3         # 'aaa'
'a' xx 3        # 'a', 'a', 'a'
\end{verbatim}
\end{frame}

\begin{frame}[fragile]
\frametitle{Equality operators}
\begin{verbatim}
"ab"    eq      "ab"    # True
"1.0"   eq      "1"     # False
"a"     ==      "b"     # True
"1"     ==      1.0     # True
1       ===     1       # True
[1, 2]  ===     [1, 2]  # False
$x = [1, 2];
$x      ===     $x      # True
$x      eqv     $x      # True
[1, 2]  eqv     [1, 2]  # True
1.0     eqv     1       # False

'abc'   ~~      m/a/    # True
1       ~~      0..4    # True
-3      ~~      0..4    # False
\end{verbatim}
\end{frame}

\begin{frame}[fragile]
\frametitle{Deref arrow is a dot nowadays}
\begin{verbatim}
# $obj->method()  # Perl 5

  $obj.method()   # Perl 6

  for @parcels {
      .address;
      .weigh;
      .ship;
      while .shipping {
          .fold;
          .spindle;
          .mutilate;
      }
      .deliver;
  }
\end{verbatim}
\end{frame}

\begin{frame}[fragile]
\frametitle{Product and zipping operators}
\begin{verbatim}
<a b> X 1..3   # ('a', 1), ('a', 2), ('a', 3),
               # ('b', 1), ('b', 2), ('b', 3)

<a b c> Z 1..3 # 'a', 1, 'b', 2, 'c', 3
\end{verbatim}
\end{frame}

\begin{frame}[fragile]
\frametitle{Two dots and three dots}
\begin{verbatim}
my $range = 1..1000;
say $range.max;      # 1000
$range = 1..*;
say $range.max;      # Whatever()<0xcba660>

@fib = 1,1 ... &[+]  # The entire Fibonacci sequence

sub foo() { ... }
\end{verbatim}
\end{frame}

\begin{frame}[fragile]
\frametitle{Foreach loops rolled into operators}
\begin{verbatim}
say [+] 1..5;       # sum
say [*] 1..5;       # product
say [<] @list;      # is @list in strict ascending order?
say [!=] @list;     # are no two consecutive items equal?
say [eq] @list;     # are all items string-equal?
say [max] @list;    # largest element

say @a >>+<< @b;      # element-wise sum
say [+] @a >>*<< @b;  # dot product from linear algebra
\end{verbatim}
\end{frame}

\begin{frame}[fragile]
\frametitle{No parentheses}
\begin{verbatim}
if $percent > 100  {
    say "weird mathematics";
}

for 1..3 {
    # using $_ as loop variable
    say 2 * $_;
}

while $stuff.is_wrong {
    $stuff.try_to_make_right;
}
\end{verbatim}
\end{frame}

\begin{frame}[fragile]
\frametitle{Powerful looping constructs}
\begin{verbatim}
    for 1..10 -> $x {
        say $x;
    }

    for 0..5 -> $even, $odd {
        say "Even: $even \t Odd: $odd";
    }

    my %h = a => 1, b => 2, c => 3;
    for %h.kv -> $key, $value {
        say "$key: $value";
    }

    for (1..1000).pick(50).kv -> $index, $value {
        say "Number $index is $value";
    }
\end{verbatim}
\end{frame}

\begin{frame}[fragile]
\frametitle{Gradual typing}
\begin{verbatim}
my Int $x = 3;
$x = "foo";         # error
say $x.WHAT;        # 'Int'

# check for a type:
if $x ~~ Int {
    say '$x contains an Int'
}

'a string'      # Str
2               # Int
3.14            # Num
(1, 2, 3)       # List
\end{verbatim}
\end{frame}

\begin{frame}[fragile]
\frametitle{Sigil invariance}
\begin{verbatim}
my $five = 5;
print "an interpolating string, just as in perl $five\n";

my @array = 1, 2, 3, 'foo';
my $sum = @array[0] + @array[1];
if $sum > @array[2] {
    say "not executed";
}
my $number_of_elems = @array.elems;     # or +@array
my $last_item = @array[*-1];

my %hash = foo => 1, bar => 2, baz => 3;
say %hash{'bar'};          # 2
say %hash<bar>;            # this is auto-quoting
\end{verbatim}
\end{frame}

\begin{frame}[fragile]
\frametitle{More no parens, \texttt{.perl}, and \texttt{.invert}}
\begin{verbatim}
my %comments =
    perl6 => <Amazing Revolutionary>,
    perl5 => <Essential Amazing>,
    perl4 => <Classic>,
    perl1 => <Classic>;

my %epithets;
%epithets.push(%comments.invert);

say %comments.perl;
# {"perl6" => ["Amazing", "Revolutionary"],
#"perl5" => ["Essential", "Amazing"], "perl4" =>
# "Classic", "perl1" => "Classic"}
say %epithets.perl;
# {"Amazing" => ["perl6", "perl5"], "Revolutionary"
# => "perl6", "Essential" => "perl5", "Classic" =>
# ["perl4", "perl1"]}
\end{verbatim}
\end{frame}

\begin{frame}[fragile]
\frametitle{Junctions}
\begin{verbatim}
if $dice_sum == 7 | 11 {
    say 'Natural!'
}
elsif $dice_sum == 2 | 3 | 12 {
    say 'Craps!'
}
\end{verbatim}
\end{frame}

\begin{frame}[fragile]
\frametitle{Chained comparisons}
\begin{verbatim}
if 20 < $bob.age < 30 {
    say 'Bob is a twenty-something.';
}

if 0 <= $angle < 2 * pi { ... }

if 0 < all(@coefficients) <= 1 {
    say 'Coefficients are already normalized.';
}
\end{verbatim}
\end{frame}

\begin{frame}[fragile]
\frametitle{Signatures}
\begin{verbatim}
# sub without a signature - perl 5 like
sub print_arguments {
    say "Arguments:";
    for @_ {
        say "\t$_";
    }
}

# Signature with fixed arity and type:
sub distance(Num $x1, Num $y1, Num $x2, Num $y2) {
    return sqrt ($x2-$x1)**2 + ($y2-$y1)**2;
}
say distance(3, 5, 0, 1); 
\end{verbatim}
\end{frame}

\begin{frame}[fragile]
\frametitle{Signatures}
\begin{verbatim}
# Default arguments
sub logarithm($num, $base = 2.7183) {
    return log($num) / log($base)
}
say logarithm(4);       # uses default second argument
say logarithm(4, 2);    # explicit second argument

# named arguments

sub doit(:$when, :$what) {
    say "doing $what at $when";
}
doit(what => 'stuff', when => 'once');
doit(:when<noon>, :what('more stuff'));
\end{verbatim}
\end{frame}

\begin{frame}[fragile]
\frametitle{Multisubs}
\begin{verbatim}
multi sub fib (Int $n where 0|1) { return $n }
multi sub fib (Int $n) { return fib($n-1) + fib($n-2) }

# or simply...

multi sub fib (0) { return 0 }
multi sub fib (1) { return 1 }
multi sub fib (Int $n) { return fib($n-1) + fib($n-2) }
\end{verbatim}
\end{frame}

\begin{frame}[fragile]
\frametitle{Operator overloading}
\begin{verbatim}
sub postfix:<!>(Int $x) {
    [*] 1..$x
}

say 5!;         # 120
\end{verbatim}
\end{frame}

\begin{frame}[fragile]
\frametitle{Unicode!}
\begin{verbatim}
\end{verbatim}
\end{frame}

\section{Object orientation}

\begin{frame}[fragile]
\frametitle{Classes}
\begin{verbatim}
class Foo {
}

class Bar is Foo { 
    # class Bar inherits from class Foo
    ...
}
\end{verbatim}
\end{frame}

\begin{frame}[fragile]
\frametitle{Methods}
\begin{verbatim}
class SomeClass {
    # these two methods do nothing but return
    # the invocant (self)
    method foo {
        return self;
    }
    method bar($s: ) {
        return $s;
    }
}

my SomeClass $x .= new;
$x.foo.bar                      # same as $x
\end{verbatim}
\end{frame}

\begin{frame}[fragile]
\frametitle{Submethods}
\begin{verbatim}
class BaseClass {
    method foo { ... }
    submethod bar { ... }
    sub baz { ... }
}

class DerivingClass is BaseClass {
}

my DerivingClass $d .= new;
$d.foo();     # works
$d.bar();     # nope, submethod
$d.baz();     # nope, sub
\end{verbatim}
\end{frame}

\begin{frame}[fragile]
\frametitle{Attributes}
\begin{verbatim}
class SomeClass {
    has $!a;
    has $.b;
    has $.c is rw;

    method do_stuff {
        return $!a + $!b + $!c;
    }
}
my $x = SomeClass.new;
say $x.a;       # ERROR!
say $x.b;       # ok
$x.b = 2;       # ERROR!
$x.c = 3;       # ok
\end{verbatim}
\end{frame}

\begin{frame}[fragile]
\frametitle{Roles}
\begin{verbatim}
role Paintable {
    has $.colour is rw;
    method paint { ... }
}
class Shape {
    method area { ... }
}

class Rectangle is Shape does Paintable {
    has $.width;
    has $.height;
    method area {
        $!width * $!height;
    }
}
\end{verbatim}
\end{frame}


\begin{frame}[fragile]
\frametitle{Enums}
\begin{verbatim}
enum Day <Mon Tue Wed Thu Fri Sat Sun>;
if custom_get_date().Day == Day::Sat | Day::Sun {
    say "Weekend";
}
\end{verbatim}
\end{frame}

\begin{frame}[fragile]
\frametitle{Subset types}
\begin{verbatim}
subset Even of Int where { $_ % 2 == 0 }
# Even can now be used like every other type name

my Even $x = 2;
my Even $y = 3; # type missmatch error
\end{verbatim}
\end{frame}

\begin{frame}[fragile]
\frametitle{Subset types}
\begin{verbatim}
sub foo (Int where { ... } $x) { ... }
# or with the variable at the front:
sub foo ($x of Int where { ... } ) { ... }
\end{verbatim}
\end{frame}

\section{Regular expressions and grammars}

\begin{frame}[fragile]
\frametitle{New regex syntax}
\begin{verbatim}
# These things work the same:
# * Capturing (...)
# * Repetition quantifiers: *, +, and ?
# * Alternatives: |
# * Backslash escape: \
# * Minimal matching suffix: ??, *?, +? 

# You have to quote all non-alphanumerics.

# A dot . matches _any_ character.
# ^ and $ match start/end of string.
# Use ^^ and $$ for start/end of line. (/s is gone)
# \n now matches a logical newline.
# \N matches anything except a newline. (/m is gone)
\end{verbatim}
\end{frame}

\begin{frame}[fragile]
\frametitle{Bracket rationalization}
\begin{verbatim}
# (...) delimits capturing groups
# [...] delimits non-capturing groups
# {...} delimits code blocks

# <...> is used for extensible metasyntax:
#   * subrule:
#      / <sign>? <mantissa> <exponent>? /
#   * code assertion:
#      / (\d**1..3) <?{ $0 < 256 }> /
#   * character classes:
#      / <[ a..z _ ]>* /
#      / <-[a..z_]> <-alpha> /
#   * zero-width assertions:
#      <?alpha>    # match null before a letter, don't capture
#      / <?before pattern> /    # lookahead
#      / <?after pattern> /     # lookbehind
\end{verbatim}
\end{frame}

\begin{frame}[fragile]
\frametitle{Repetition}
\begin{verbatim}
. ** 42                  # match exactly 42 times
<item> ** 3..*           # match 3 or more times

<alt> ** '|'            # separator is character
<addend> ** <addop>     # separator is subrule
<item> ** [ \!?'==' ]   # separator is operator
<file>**\h+             # separator is whitespace
\end{verbatim}
\end{frame}

\begin{frame}[fragile]
\frametitle{Backtracking}
\begin{verbatim}
# :        make preceeding atom not backtrack
# ::       fail surrounding group instead of backtracking
# :::      fail current regex     instead of backtracking
# commit   fail entire match      instead of backtracking
# cut      continue, but delete everything up to here
\end{verbatim}
\end{frame}

\begin{frame}[fragile]
\frametitle{Regex, token, rule}
\begin{verbatim}
regex { pattern }    # always {...} as delimiters
rx    / pattern /    # (almost) any chars as delimiters

token ident { [ <alpha> | _ ] \w* }     # same as...
regex ident { [ <alpha>: | _: ]: \w*: } # ...this

rule quux { next cmd '='   <condition> }    # same as...
token quux { <.ws> next <.ws> cmd <.ws> '=' # ...this
             <.ws> <condition> }
\end{verbatim}
\end{frame}

\begin{frame}[fragile]
\frametitle{Grammars}
\begin{verbatim}
grammar XML {
    token TOP   { ^ <xml> $ };
    token xml   { <text> [ <tag> <text> ]* };
    token text {  <-[<>&]>* };
    rule tag   {
        '<'(\w+) <attributes>*
        [
            | '/>'                 # a single tag
            | '>'<xml>'</' $0 '>'  # an opening
                                   # and a closing tag
        ]
    };
    token attributes { \w+ '="' <-["<>]>* '"' };
};
\end{verbatim}
\end{frame}

\begin{frame}[fragile]
\frametitle{Grammar inheritance}
\begin{verbatim}
grammar Letter {
    rule text  { <greet> <body> <close> }
    rule greet { [Hi|Hey|Yo] $<to>=(\S+?) , $$}
    rule body  { <line>+? }
    rule close { Later dude, $<from>=(.+) }
    # etc.
}

grammar FormalLetter is Letter {
    rule greet { Dear $<to>=(\S+?) , $$}
    rule close { Yours sincerely, $<from>=(.+) }
}
\end{verbatim}
\end{frame}

\begin{frame}[fragile]
\frametitle{Actions}
\begin{verbatim}
grammar Integer {
    token TOP {
        | 0b<[01]>+ {*}  #= binary
        | \d+       {*}  #= decimal
    }
}
class Twice {
    multi method TOP($match, $tag) {
        my $text = ~$match;
        $text = :2($text) if $tag eq 'binary'
        make $text;
    }
    multi method TOP($match) {
        make 2 * $match.ast;
    }
}
Integer.parse('21', :action(Twice.new)).ast      # 42
\end{verbatim}
\end{frame}

\begin{frame}[fragile]
\frametitle{And more...}
\begin{verbatim}
given $source_code {
    $parsetree = m:keepall/<Perl::prog>/;
}

grammar LolPerl is STD { ... }

$~MAIN = LolPerl;

# LOL!
\end{verbatim}
\end{frame}

\begin{frame}[fragile]
\frametitle{Bibliography}
\begin{verbatim}
The material in this talk has been nicked^Wadapted from
a number of sources.
\end{verbatim}
\begin{thebibliography}{10} 
\bibitem{moritz}
Moritz Lentz' "Perl 5 to Perl 6" tutorial.
\bibitem{conway}
An email from Damian Conway to perl6-language.
\bibitem{wall}
The synopses.
\end{thebibliography} 
\end{frame}

\begin{frame}[fragile]
\frametitle{KTHXBAI}
\begin{verbatim}
# Questions?

# ("When will Perl 6 be released?")
\end{verbatim}
\end{frame}

\end{document}

